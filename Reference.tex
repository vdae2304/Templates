\documentclass[10pt, letterpaper, twoside]{article}
\usepackage[utf8]{inputenc}
\usepackage[spanish]{babel}
\usepackage[left = 2cm, right = 2cm, top = 2cm, bottom = 2cm]{geometry}

\usepackage{amsmath, amsfonts, amssymb, amsthm}

\title{Material de referencia para la ICPC.}
\author{}
\date{} 

%Estilo de la página
\usepackage{fancybox, fancyhdr}
\pagestyle{fancy}
\fancyhf{}
\fancyhead[LE,RO]{\small{\leftmark}}
\fancyfoot[CE,CO]{\thepage}
\renewcommand{\headrulewidth}{2 pt}

%Estilo del codigo
\usepackage{listings}
\usepackage[dvipsnames]{xcolor}
\lstset{  
	language         = C++, 
	xleftmargin      = 1 cm,
	numbers          = left,
	numberstyle      = \tiny\textbf,	
	basicstyle       = \footnotesize,
	keywordstyle     = \color{blue},
	directivestyle   = \color{Green},
	commentstyle     = \color{purple},
	stringstyle      = \color{blue},
	showstringspaces = false,
	breaklines       = true,
}

%Links en el documento.
\usepackage{hyperref}
\hypersetup{
	bookmarks  = true,	
	pdfstartpage = 1,
	pdftitle = "ICPC Reference",
	linkbordercolor	= 1 1 1,
}

%Documento
\begin{document}

\maketitle

\tableofcontents

\newpage

\section{Estructuras de datos.}

\subsection{Treap.}

Complejidad promedio por operación: $O(\log n)$

\lstinputlisting[firstline = 9, lastline = 93]{Estructuras/Treap.cpp} \medskip

%\newpage

\section{Grafos.}

\subsection{Caminos más cortos.}

\textbf{Algoritmo de Dijkstra.} Complejidad: $O((E + V) \log V)$.

\lstinputlisting[firstline = 6]{Grafos/Dijkstra.cpp}

\subsection{Árbol de expansión mínima.}

\textbf{Algoritmo de Kruskal.} Complejidad: $O(E \log V)$.

\lstinputlisting[firstline = 6]{Grafos/Kruskal.cpp}

\subsection{Orden topológico.}

Complejidad: $O(V + E)$.

\lstinputlisting[firstline = 6]{Grafos/Topological-Sort.cpp}

\subsection{Componentes fuertemente conexas.}

\textbf{Algoritmo de Kosaraju.} Complejidad: $O(V + E)$.

\lstinputlisting[firstline = 6]{Grafos/Kosaraju.cpp}

\subsection{Puentes y puntos de articulación.}

\textbf{Algoritmo de Tarjan.} Complejidad: $O(V + E)$.

\lstinputlisting[firstline = 6]{Grafos/Bridge-Articulation.cpp}

\subsection{Flujo máximo.}

\textbf{Algoritmo de Dinic.} Complejidad: $O(V^2 E)$.

\lstinputlisting[firstline = 6]{Grafos/Dinic.cpp}

\subsection{Emparejamiento máximo.}

\textbf{Algoritmo de Hopcroft-Karp.} Complejidad: $O(\sqrt{V}E)$.

\lstinputlisting[firstline = 7]{Grafos/Hopcroft-Karp.cpp}

%\newpage

\section{Matemáticas.}

\subsection{Big Numbers.}

\lstinputlisting[firstline = 7, lastline = 123]{Matematicas/Big-Numbers.cpp}

\subsection{Test de Primalidad.}

\textbf{Algoritmo de Miller-Rabin (determinista).} Complejidad: $O(\log n)$.

\lstinputlisting[firstline = 6]{Matematicas/Miller-Rabin.cpp} 

\subsection{Factorización en primos.}

Complejidad: $O\left(\pi\left(\sqrt{n}\right)\right)$ donde $\pi(x)$ es el número de primos menores o iguales que $x$.

\lstinputlisting[firstline = 6]{Matematicas/Trial-Division.cpp}

\subsection{Sistemas de Ecuaciones Lineales.}

\textbf{Eliminación Gauss-Jordan.} Complejidad $O(n^3)$.

\lstinputlisting[firstline = 6]{Matematicas/Gauss-Jordan.cpp}

\subsection{Teorema Chino del Residuo.}

\lstinputlisting[firstline = 6]{Matematicas/Chinese-Remainder.cpp}

%\newpage

\section{Geometría}

\subsection{Geometría 2D.}

\lstinputlisting[firstline = 6, lastline = 146]{Matematicas/Geometry.cpp}

\subsection{Envolvente convexa.}

\textbf{Algoritmo de Graham-Scan.} Complejidad: $O(n \log n)$.

\lstinputlisting[firstline = 6]{Matematicas/Convex-Hull.cpp}

%\newpage

\section{Strings}

\subsection{Búsqueda de patrones.}

\textbf{Algoritmo de KMP.} Complejidad: $O(|P| + |T|)$.

\lstinputlisting[firstline = 6]{Strings/KMP.cpp}

\bigskip

\textbf{Aho Corasick.} Complejidad: $O(|T| + |P_1| + \ldots + |P_n| + \#Ocurrencias)$.

\lstinputlisting[firstline = 7]{Strings/Aho-Corasick.cpp}

\subsection{Arreglo de sufijos.}

Complejidad: $O(n\log n)$.

\lstinputlisting[firstline = 6]{Strings/Suffix-Array.cpp}

%\newpage

\end{document}