\documentclass[12pt, letterpaper, twoside]{article}
\usepackage[utf8]{inputenc}
\usepackage[spanish]{babel}
\usepackage{amsmath, amsfonts, amssymb, amsthm}
\usepackage[left = 2cm, right = 2cm, top = 2cm, bottom = 2cm]{geometry}

\title{Matemáticas.\\
	  \large Material de referencia.}
\author{}
\date{} 

%Estilo de la página
\usepackage{fancybox, fancyhdr}
\pagestyle{fancy}
\fancyhf{}
\fancyhead[LE,RO]{\small{\leftmark}}
\fancyfoot[CE,CO]{\thepage}
\renewcommand{\headrulewidth}{2 pt}

%Imagenes
\usepackage{graphicx}
\graphicspath{{Imagenes/}}

%Estilo del codigo
\usepackage{listings}
\usepackage[dvipsnames]{xcolor}
\lstset{  
	language         = C++, 
	xleftmargin      = 1 cm,
	numbers          = left,
	numberstyle      = \tiny\textbf,	
	basicstyle       = \footnotesize,
	keywordstyle     = \color{blue},
	directivestyle   = \color{Green},
	commentstyle     = \color{purple},
	stringstyle      = \color{blue},
	showstringspaces = false,
	breaklines       = true,
}

%Documento
\begin{document}

\maketitle

\tableofcontents

\newpage

\section{Fórmulas importantes.}

\subsection{Desarreglos}

Un desarreglo es una permutación donde ningún elemento aparece en su posición original. El número de desarreglos está dado por la fórmula recursiva 
$$!n = (n - 1)(!(n - 1) + !(n - 2)), \qquad !0 = 1, \, !1 = 0$$ 
y por la fórmula cerrada
$$!n = n! \sum_{k=0}^n \frac{(-1)^k}{k!}.$$

\subsection{Números de Catalán}

Los números de Catalán cuentan: el número de expresiones con $n$ pares de paréntesis correctamente balanceados; el número de caminos distintos sobre una cuadrícula de $n \times n$ que empiezan en la esquina inferior izquierda y terminan en la esquina superior derecha, constan solamente de movimientos hacia arriba y hacia la derecha, y nunca cruzan la diagonal; el número de triangulaciones de un polígono convexo de $n + 2$ lados; entre otras cosas. Están dados por la fórmula recursiva
$$C_{n+1} = \sum_{i=0}^n C_iC_{n-i}, \qquad C_0 = 1$$
y por la fórmula cerrada
$$C_n = \frac{1}{n + 1}\binom{2n}{n}.$$

\subsection{Números de Stirling}

Los números de Stirling de primer tipo cuentan el número de permutaciones con exactamente $k$ ciclos disjuntos. Están dados por la fórmula recursiva
$$c(n + 1, k) = nc(n, k) + c(n, k - 1), \qquad c(n, 0) =  0, \, c(n, n) = 1.$$

Los números de Stirling de segundo tipo cuentan el número de particiones de un conjunto de tamaño $n$ en $k$ subconjuntos no vacíos. Están dados por la fórmula recursiva
$$S(n + 1, k) = kS(n, k) + S(n, k - 1), \qquad S(n, 0) =  0, \, S(n, n) = 1$$
y por la fórmula cerrada
$$S(n, k) = \frac{1}{k!} \sum_{i=0}^k (-1)^i \binom{k}{i}(k - i)^n.$$

\subsection{Números de Grundy}

Un juego por turnos entre dos jugadores es \textit{normal} si el jugador que no pueda mover pierde, y es \textit{imparcial} si en todo momento ambos jugadores disponen del mismo conjunto de movimientos.

El juego de \textit{Nim} es un juego normal e imparcial en donde cada jugador debe escoger una pila y eliminar al menos un objeto de esa pila. Sean $P_1, \ldots, P_n$ los tamaños de cada pila. El jugador en turno tiene estrategia ganadora si y sólo si $P_1 \text{ xor } \ldots \text{ xor } P_n \neq 0$.

El \textbf{Teorema de Sprague-Grundy} afirma que todo juego normal e imparcial es equivalente a un juego de Nim.

\newpage

\section{Big Numbers.}

\subsection{Implementación}

\lstinputlisting[firstline = 6]{BigNumbers.cpp} \medskip

\begin{tabular}{|p{7cm}|p{7cm}|}
\hline
\textbf{Entrada} & \textbf{Salida}\\ \hline
1894821 & 1894821 + 589613 = 2484434\\
589613  & 1894821 - 589613 = 1305208\\ 
        & 1894821 * 589613 = 1117211094273\\ 
        & 1894821 = 589613 * 3 + 125982\\ \hline
\end{tabular}

\newpage

\section{Test de Primalidad.}

Decimos que un entero positivo $p$ es primo si tiene exactamente dos divisores distintos: $1$ y $p$.

\subsection{Algoritmo de Miller-Rabin.}

\lstinputlisting[firstline = 6]{Miller-Rabin.cpp} \medskip

\begin{tabular}{|p{7cm}|p{7cm}|}
\hline
\textbf{Entrada} & \textbf{Salida}\\ \hline
1000000007 & Probablemente es primo.\\
123456789  & No es primo.\\
104729     & Probablemente es primo.\\ \hline
\end{tabular}

\newpage

\section{Factorización en primos.}

Sea $n$ un entero mayor que $1$, el Teorema Fundamental de la Aritmética afirma que $n$ tiene una única factorización en primos.

\subsection{Algoritmo de prueba por división.}

Complejidad: $O\left(\pi\left(\sqrt{n}\right)\right)$ donde $\pi(x)$ es el número de primos menores o iguales que $x$.

\lstinputlisting[firstline = 6]{Trial-Division.cpp} \medskip

\begin{tabular}{|p{7cm}|p{7cm}|}
\hline
\textbf{Entrada} & \textbf{Salida}\\ \hline
180   & 2 2 3 3 5\\
3500  & 2 2 5 5 5 7\\ 
123456789 & 3 3 3607 3803\\
104729 & 104729\\ \hline
\end{tabular}

\newpage

\section{Sistema de Ecuaciones Lineales.}

Consideremos un sistema de ecuaciones lineales dado por la matriz \textbf{A} de $n \times n$ y el vector \textbf{b} de dimensión $n$. Decimos que \textbf{x} es solución si $\textbf{Ax} = \textbf{b}$.

\subsection{Eliminación Gaussiana}

Complejidad $O(n^3)$.

\lstinputlisting[firstline = 6]{GaussianElimination.cpp} \medskip

\begin{tabular}{|p{7cm}|p{7cm}|}
\hline
\textbf{Entrada} & \textbf{Salida}\\ \hline
4 1          & Determinante: 142\\
1 -2 2 -3 15 & Solucion:\\ 
3 4 -1 1 -6  & 2\\ 
2 -3 2 -1 17 & -2\\ 
1 1 -3 -2 -7 & 3\\ 
             & -1\\ \hline
\end{tabular}

\newpage

\section{Sistema de Congruencias Lineales.}

Consideremos el sistema de congruencias
\begin{align*}
x &\equiv a_1 \pmod{m_1}\\
&\vdots\\
x &\equiv a_n \pmod{m_n}
\end{align*}
con $m_1, \ldots, m_n$ primos relativos por parejas. El Teorema Chino del Residuo afirma que existe una única solución módulo $m_1\cdots m_n$.

\subsection{Teorema Chino del Residuo.}

\lstinputlisting[firstline = 6]{ChineseRemainder.cpp} \medskip

\begin{tabular}{|p{7cm}|p{7cm}|}
\hline
\textbf{Entrada} & \textbf{Salida}\\ \hline
3   & x = 66 (mod 180)\\
2 4 & \\ 
3 9 & \\ 
1 5 & \\ \hline
\end{tabular}

\newpage

\end{document}