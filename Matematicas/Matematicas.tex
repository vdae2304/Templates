\documentclass[12pt, letterpaper, twoside]{article}
\usepackage[utf8]{inputenc}
\usepackage[spanish]{babel}
\usepackage{amsmath, amsfonts, amssymb, amsthm}
\usepackage[left = 2cm, right = 2cm, top = 2cm, bottom = 2cm]{geometry}

\title{Matemáticas.\\
	  \large Material de referencia.}
\author{}
\date{} 

%Estilo de la página
\usepackage{fancybox, fancyhdr}
\pagestyle{fancy}
\fancyhf{}
\fancyhead[LE,RO]{\small{\leftmark}}
\fancyfoot[CE,CO]{\thepage}
\renewcommand{\headrulewidth}{2 pt}

%Imagenes
\usepackage{graphicx}
\graphicspath{{Imagenes/}}

%Estilo del codigo
\usepackage{listings}
\usepackage[dvipsnames]{xcolor}
\lstset{  
	language         = C++, 
	xleftmargin      = 1 cm,
	numbers          = left,
	numberstyle      = \tiny\textbf,	
	basicstyle       = \footnotesize,
	keywordstyle     = \color{blue},
	directivestyle   = \color{Green},
	commentstyle     = \color{purple},
	stringstyle      = \color{blue},
	showstringspaces = false,
	breaklines       = true,
}

%Documento
\begin{document}

\maketitle

\tableofcontents

\newpage

\section{Big Numbers.}

\subsection{Implementación}

\lstinputlisting[firstline = 6]{BigNumbers.cpp} \medskip

\begin{tabular}{|p{7cm}|p{7cm}|}
\hline
\textbf{Entrada} & \textbf{Salida}\\ \hline
1894821 & 1894821 + 589613 = 2484434\\
589613  & 1894821 - 589613 = 1305208\\ 
        & 1894821 * 589613 = 1117211094273\\ 
        & 1894821 = 589613 * 3 + 125982\\ \hline
\end{tabular}

\newpage

\section{Test de Primalidad.}

Decimos que un entero positivo $p$ es primo si tiene exactamente dos divisores distintos: $1$ y $p$.

\subsection{Algoritmo de Miller-Rabin.}

\lstinputlisting[firstline = 6]{Miller-Rabin.cpp} \medskip

\begin{tabular}{|p{7cm}|p{7cm}|}
\hline
\textbf{Entrada} & \textbf{Salida}\\ \hline
1000000007 & Probablemente es primo.\\
123456789  & No es primo.\\
104729     & Probablemente es primo.\\ \hline
\end{tabular}

\newpage

\section{Factorización en primos.}

Sea $n$ un entero mayor que $1$, el Teorema Fundamental de la Aritmética afirma que $n$ tiene una única factorización en primos.

\subsection{Algoritmo de prueba por división.}

Complejidad: $O\left(\pi\left(\sqrt{n}\right)\right)$ donde $\pi(x)$ es el número de primos menores o iguales que $x$.

\lstinputlisting[firstline = 6]{Trial-Division.cpp} \medskip

\begin{tabular}{|p{7cm}|p{7cm}|}
\hline
\textbf{Entrada} & \textbf{Salida}\\ \hline
180   & 2 2 3 3 5\\
3500  & 2 2 5 5 5 7\\ 
123456789 & 3 3 3607 3803\\
104729 & 104729\\ \hline
\end{tabular}

\newpage

\section{Sistema de Ecuaciones Lineales.}

Consideremos un sistema de ecuaciones lineales dado por la matriz \textbf{A} de $n \times n$ y el vector \textbf{b} de dimensión $n$. Decimos que \textbf{x} es solución si $\textbf{Ax} = \textbf{b}$.

\subsection{Eliminación Gaussiana}

Complejidad $O(n^3)$.

\lstinputlisting[firstline = 6]{GaussianElimination.cpp} \medskip

\begin{tabular}{|p{7cm}|p{7cm}|}
\hline
\textbf{Entrada} & \textbf{Salida}\\ \hline
4 1          & Determinante: 142\\
1 -2 2 -3 15 & Solucion:\\ 
3 4 -1 1 -6  & 2\\ 
2 -3 2 -1 17 & -2\\ 
1 1 -3 -2 -7 & 3\\ 
             & -1\\ \hline
\end{tabular}

\newpage

\section{Sistema de Congruencias Lineales.}

Consideremos el sistema de congruencias
\begin{align*}
x &\equiv a_1 \pmod{m_1}\\
&\vdots\\
x &\equiv a_n \pmod{m_n}
\end{align*}
con $m_1, \ldots, m_n$ primos relativos por parejas. El Teorema Chino del Residuo afirma que existe una única solución módulo $m_1\cdots m_n$.

\subsection{Teorema Chino del Residuo.}

\lstinputlisting[firstline = 6]{ChineseRemainder.cpp} \medskip

\begin{tabular}{|p{7cm}|p{7cm}|}
\hline
\textbf{Entrada} & \textbf{Salida}\\ \hline
3   & x = 66 (mod 180)\\
2 4 & \\ 
3 9 & \\ 
1 5 & \\ \hline
\end{tabular}

\newpage

\end{document}