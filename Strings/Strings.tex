\documentclass[12pt, letterpaper, twoside]{article}
\usepackage[utf8]{inputenc}
\usepackage[spanish]{babel}
\usepackage{amsmath, amsfonts, amssymb, amsthm}
\usepackage[left = 2cm, right = 2cm, top = 2cm, bottom = 2cm]{geometry}

\title{Strings.\\
	  \large Material de referencia.}
\author{}
\date{} 

%Estilo de la página
\usepackage{fancybox, fancyhdr}
\pagestyle{fancy}
\fancyhf{}
\fancyhead[LE,RO]{\small{\leftmark}}
\fancyfoot[CE,CO]{\thepage}
\renewcommand{\headrulewidth}{2 pt}

%Imagenes
\usepackage{graphicx}
\graphicspath{{Imagenes/}}

%Estilo del codigo
\usepackage{listings}
\usepackage[dvipsnames]{xcolor}
\lstset{  
	language         = C++, 
	xleftmargin      = 1 cm,
	numbers          = left,
	numberstyle      = \tiny\textbf,	
	basicstyle       = \footnotesize,
	keywordstyle     = \color{blue},
	directivestyle   = \color{Green},
	commentstyle     = \color{purple},
	stringstyle      = \color{blue},
	showstringspaces = false,
	breaklines       = true,
}

%Documento
\begin{document}

\maketitle

\tableofcontents

\newpage

\section{Arreglo de sufijos.}

Consideremos un string $s$. Un arreglo de sufijos de $s$ es un arreglo de enteros que guarda las posiciones iniciales de los sufijos de $s$ en orden lexicográfico.

\subsection{Implementación}

Complejidad: $O(|s|\log |s|)$.

\lstinputlisting[firstline = 6]{Suffix-Array.cpp} \medskip

\begin{tabular}{|p{7cm}|p{7cm}|}
\hline
\textbf{Entrada} & \textbf{Salida}\\ \hline
banana & 5 a\\
       & 3 ana\\
       & 1 anana\\
       & 0 banana\\
       & 4 na\\
       & 2 nana\\ \hline
\end{tabular}

\newpage

\end{document}